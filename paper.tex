\documentclass[twoside]{article}

%\usepackage{aistats2020}
% If your paper is accepted, change the options for the package
% aistats2020 as follows:
%
\usepackage[accepted]{aistats2020}
%
% This option will print headings for the title of your paper and
% headings for the authors names, plus a copyright note at the end of
% the first column of the first page.

% If you set papersize explicitly, activate the following three lines:
%\special{papersize = 8.5in, 11in}
%\setlength{\pdfpageheight}{11in}
%\setlength{\pdfpagewidth}{8.5in}

% If you use natbib package, activate the following three lines:
\usepackage[round]{natbib}
\renewcommand{\bibname}{References}
\renewcommand{\bibsection}{\subsubsection*{\bibname}}

% If you use BibTeX in apalike style, activate the following line:
\bibliographystyle{apalike}

% math and stuff
\usepackage{amssymb}
\usepackage{amsmath}
\usepackage{amsthm}
\usepackage{algorithmicx}
\usepackage{algpseudocode}
%\newcommand{\disc}{\mathrm{disc}}
%\newcommand{\cont}{\mathrm{cont}}
%\newcommand{\tr}{\mathtt{tr}}
%\newcommand{\model}{\mathcal{P}}
%\newcommand{\proposal}{\mathcal{Q}}

\begin{document}

% If your paper is accepted and the title of your paper is very long,
% the style will print as headings an error message. Use the following
% command to supply a shorter title of your paper so that it can be
% used as headings.
%
%\runningtitle{I use this title instead because the last one was very long}

% If your paper is accepted and the number of authors is large, the
% style will print as headings an error message. Use the following
% command to supply a shorter version of the authors names so that
% they can be used as headings (for example, use only the surnames)
%
%\runningauthor{Surname 1, Surname 2, Surname 3, ...., Surname n}

\twocolumn[

\aistatstitle{Automated Involutive MCMC}

\aistatsauthor{ Marco Cusumano-Towner \And Alexander K. Lew \And Vikash K. Mansinghka }

\aistatsaddress{ Massachusetts Institute of Technology } ]

\begin{abstract}
  The Abstract paragraph should be indented 0.25 inch (1.5 picas) on
  both left and right-hand margins. Use 10~point type, with a vertical
  spacing of 11~points. The \textbf{Abstract} heading must be centered,
  bold, and in point size 12. Two line spaces precede the
  Abstract. The Abstract must be limited to one paragraph.
\end{abstract}

\section{INTRODUCTION}

This is the best paper~\citep{cusumano2019gen}.

\subsection{Second Level Heading}

\subsubsection{Third Level Heading}

\paragraph{Fourth Level Heading}

\section{BACKGROUND AND RELATED WORK}

\subsection{Traces}

\section{INVOLUTIVE MCMC WITH TRACES}

\begin{algorithmic}
\Require Target distribution $p$, proposal distribution $q$, involution $h(u, t)$, previous state $x$.
\Procedure{involutive-mcmc-move}{$p$, $q$, $f$, $x$}
    \State $u \sim q(\cdot; x)$ \Comment{Sample auxiliary variables}
    \State $(x', u') \gets f(x, u)$ \Comment{Apply involution}
    \State $\alpha \gets
        \frac{p(x') q(u'; x')}{p(x) q(u; x)} \left| \left[ \frac{\partial h(u, t)}{\partial (u, t)} \right] \right|$
    \State $r \sim \mathrm{Uniform}(0, 1)$
    \State \algorithmicif \, $r \le \alpha$ \algorithmicthen \, \Return $x'$ \algorithmicelse \, \Return $x$ 
\EndProcedure
\end{algorithmic}

% TODO PyTorch implementation??

\section{AUTOMATIC ACCEPTANCE PROBABILITY CALCULATION}

\subsection{Exploiting Jacobian Sparsity}

\subsection{Exploiting Jacobian Sparsity}

\subsection{Exploiting Jacobian Sparsity}

\section{A DYNAMIC CHECK FOR DETECTING BUGS}

\section{EXAMPLES}


\subsubsection*{Acknowledgements}
All acknowledgments go at the end of the paper, including thanks to reviewers who gave useful comments, to colleagues who contributed to the ideas, and to funding agencies and corporate sponsors that provided financial support.

\bibliography{references} 

\clearpage
\onecolumn
\section*{APPENDIX}

\subsection{Formalism}
Let $(Z, \Sigma, \mu)$ be a measure space of pairs of traces of the model and proposal, where $\mu$ is $\sigma$-finite.
Consider a $\mu$-measurable function $h : Z \to Z$ that is an involution ($h(h(z)) = z$) and let $\mu \circ h^{-1}$ denote the pushforward of $\mu$ under $h$, such that $\mu \circ h^{-1}$ is $\sigma$-finite and such that $\mu \circ h^{-1}$ is absolutely continuous with respect to $\mu$.
% TODO can we simplify the conditions, using the fact that h is an involution?
Then, the Radon-Nikodym derivative exists, and is denoted $(d (\mu \circ h^{-1}) / d \mu)(z)$.

The goal is to prove stationarity with respect to the measure defined by density $p(z)$ with respect to $\mu$.
The move starts with state $z \in Z$, applies the involution $h$, and then accepts (returns $h(z)$ with probability $\alpha(z)$) and otherwise returns $z$.
Let $k_{z}$ denote the measure on new states induced by the involution $h$ and the previous state $z$:
\begin{equation}
k_{z}(A) = [h(z) \in A] \alpha(z) + [z \in A] (1 - \alpha(z))
\end{equation}
The stationarity condition is:
\begin{equation}
\int_Z p(z) k_{z}(A) \mu(dz) = p(A) = \int_Z p(z) [z \in A] \mu(dz) \mbox{ for all } A \in \Sigma
\end{equation}
Substituting in the definition of $k_z(A)$ and simplifying gives:
\begin{equation} \label{eq:stationarity-requirement}
\int_Z p(z) [h(z) \in A] \alpha(z) \mu(dz) = \int_Z p(z) [z \in A] \alpha(z) \mu(dz) \mbox{ for all } A \in \Sigma
\end{equation}
Note that for a $\mu$-measurable function $g$ such that $g \circ h$ is integrable with respect to $\mu$:
\begin{equation} \label{eq:pushforward}
\int_Z g(h(z)) \mu(dz) = \int_Z g(z) (\mu \circ h^{-1})(dz) = \int_Z g(z) \left( \frac{d (\mu \circ h^{-1})}{d \mu}(z) \right) \mu(dz)
\end{equation}
For $g(z) := p(h(z)) [z \in A] \alpha(h(z))$ we have $g(h(z)) = p(z) [h(z) \in A] \alpha(z)$, which is the integrand of the left-hand-side of Equation~(\ref{eq:stationarity-requirement}).
Applying Equation~(\ref{eq:pushforward}) to the left-hand side of Equation~(\ref{eq:stationarity-requirement}) gives:
\begin{equation}
\int_Z p(z) [h(z) \in A] \alpha(z) \mu(dz) = \int_Z g(z) \left( \frac{d (\mu \circ h^{-1})}{d \mu}(z) \right) \mu(dz) = \int_Z p(h(z)) [z \in A] \alpha(h(z)) \left( \frac{d (\mu \circ h^{-1})}{d \mu}(z) \right) \mu(dz)
\end{equation}
Equating this with the right-hand side of Equation~(\ref{eq:stationarity-requirement}) gives the following sufficient condition for stationarity:
\begin{equation}
\int_Z p(h(z)) [z \in A] \alpha(h(z)) \left( \frac{d (\mu \circ h^{-1})}{d \mu}(z) \right) \mu(dz) = \int_Z p(z) [z \in A] \mu(dz) \mbox{ for all } A \in \Sigma
\end{equation}
It therefore suffices to find $\alpha$ such that:
\begin{equation}
p(h(z)) [z \in A] \alpha(h(z)) \left( \frac{d (\mu \circ h^{-1})}{d \mu}(z) \right) = p(z) [z \in A] \alpha(z) \mbox{ for all } z \in Z
\end{equation}
The following choice of $\alpha$ suffices:
\begin{equation}
\alpha(z) = \min\left\{ 1, \frac{p(h(z))}{p(z)} \left( \frac{d (\mu \circ h^{-1})}{d \mu}(z) \right) \right\}
\end{equation}

All requirements:
\begin{enumerate}
\item $\mu$ is $\sigma$-finite
\item $h$ is $\mu$-measurable
\item $h$ is an involution
\item the pushforward $\mu \circ h^{-1}$ is $\sigma$-finite
\item the pushforward $\mu \circ h^{-1}$ is absolutely continuous with respect to $\mu$
\item the function $g(z) := p(h(z)) [z \in A] \alpha(h(z))$ is $\mu$-measurable for all $A \in \Sigma$
\item the function $(g \circ h)(z) = p(z) [h(z) \in A] \alpha(z)$ is integrable with respect to $\mu$ for all $A \in \Sigma$
\end{enumerate}

\end{document}
